\documentclass[12pt]{article}

\title{Cross Product}
\date{}

\usepackage{graphicx} % LaTeX package to import graphics
\graphicspath{{Math/linear_algebra/images}} % Configuring the graphicx package

\usepackage[a4paper, margin=1.5cm, footskip=.5cm]{geometry}
\usepackage{amssymb}
\usepackage{amsmath}
\usepackage{parskip}
\usepackage{tocloft}
\usepackage{float}

\renewcommand{\cftsecleader}{\cftdotfill{\cftdotsep}} % Adds dots to TOC
\setcounter{secnumdepth}{0} % sections are level 1


% Set all sections to have centered headings
\usepackage{titlesec}
\titleformat{\section}  % which section command to format
    {\centering\fontsize{17}{19}\bfseries} % format for whole line
    {\thesection} % how to show number
    {1em} % space between number and text
    {} % formatting for just the text
    [] % formatting for after the text
% --------------------------------------------


% Macro to reduce code repetition
%
% \NewDocumentCommand{\macroname}{argument specifiers}{%
%     expansion code
% }
% 
% "m" is a standard mandatory argument, which can be a single token or multiple tokens surrounded by {}
% "o" is a standard optional argument, it will supply -NoValue- marker if not given
% "O", given as O{(default)}, is like "o" but returns default if no value is given
% "v" reads an argument ‘verbatim’

\NewDocumentCommand{\Vector}{mmo}{%
  \begin{bmatrix} #1 \\ #2 \IfValueT{#3}{\\ #3} \end{bmatrix}%
}

\NewDocumentCommand{\InlineFigure}{ O{0.70\textwidth} O{0.35\textwidth} m m v }{%
    \begin{figure}[H]
        \centering
        \includegraphics[width=#1, height=#2]{#5}
        \caption{#3}
        \label{#4}
    \end{figure}
}
\begin{document}
    \maketitle

The cross product is defined only for three-dimensional vectors. If $a$ and $b$ are two three-dimensional vectors, then their cross product, written as $a \times b$ and pronounced "$a$ cross $b$", \textbf{is another three-dimensional vector}. We define this cross product $a \times b$ by the following three requirements.

\begin{enumerate}
    \item $a \times b$ is a vector, that is \textbf{perpendicular} to both $a$ and $b$
    \item The magnitude (or length) of the vector $a \times b$, written as $\| a \times b \|$, is the area of the parallelogram spanned by $a$ and $b$
    \item The direction of $a \times b$ is determined by the right-hand rule. This means that if we curl the fingers of the right hand from $a$ to $b$, then the thumb points in the direction of $a \times b$
\end{enumerate}

The below figure illustrates how using trigonometry, we can calculate that the area of the parallelogram spanned by $a$ and $b$ is $\| a \| \| b \| \sin(\theta)$ where $\theta$ is the angle between $a$ and $b$. The figure shows the parallelogram as having a base of length $\| b \|$ and perpendicular height of $\| a \| \sin(\theta)$.

\InlineFigure{Cross Product}{fig1}{cross_product_1.png}

This formula shows that the magnitude of the cross product is largest when $a$ and $b$ are perpendicular. On the other hand, if $a$ and $b$ are parallel or if either vector is the zero vector, then the cross product is the zero vector.

\InlineFigure{Cross Product}{fig2}{cross_product_2.png}

\section{How to calculate the Cross Product}

Given two vectors $\Vec{u}$ = \Vector{$u_1$}{$u_2$}[$u_3$] and $\Vec{v}$ = \Vector{$v_1$}{$v_2$}[$v_3$], the vector product or cross product $\Vec{u} \times \Vec{v}$ can be calculated using the following;

\begin{center}
    $\Vec{u} \times \Vec{v}$ = $(u_2v_3 - v_2v_3)\Vec{\hat{i}}$ - $(u_1v_3 - v_1u_3)\Vec{\hat{j}}$ + $(u_1v_2 - v_1u_2)\Vec{\hat{k}}$

    or
    
    $\Vec{u} \times \Vec{v}$ = $(u_2v_3 - v_2v_3)\Vec{\hat{i}}$ + $(v_1u_3 - u_1v_3)\Vec{\hat{j}}$ + $(u_1v_2 - v_1u_2)\Vec{\hat{k}}$
\end{center}

\subsection{Example}

$\Vec{u}$ = \Vector{1}{-3}[2] $\Vec{v}$ = \Vector{4}{0}[6]

$\Vec{u} \times \Vec{v}$ = 
$(-3 \times 6 - 0 \times 2)\hat{i}$ +
$(4 \times 2 - 1 \times 6)\hat{j}$ +
$(1 \times 0 - 4 \times -3)\hat{k}$
$\Vec{u} \times \Vec{v}$ = $\Vec{-18\hat{i}}$ + $\Vec{2\hat{j}}$ $\Vec{12\hat{k}}$





\end{document}